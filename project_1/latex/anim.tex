\documentclass{standalone}
\usepackage{animate}
\usepackage{graphicx}
\usepackage{ifthen}
\usepackage{tikz}
\usetikzlibrary{backgrounds}

\newcommand{\nodelabel}[1]{%
    {\LARGE{\textcolor{blue}{$#1$}}}
}

\newcommand{\drawpoints}{%
    \draw (0, 0) rectangle (16, 16);
    \coordinate (A) at (5.5, 2.5);
    \coordinate (B) at (6.5, 5.5);
    \coordinate (C) at (5.5, 6.5);
    \coordinate (D) at (7.5, 7.5);
    \coordinate (E) at (9.5, 6.5);
    \coordinate (F) at (8.5, 8.5);
    \coordinate (G) at (12.5, 9.5);
    \coordinate (H) at (9.5, 12.5);
}

\newcommand{\drawquadtree}{%
    \draw (8, 0) -- (8, 16);
    \draw (0, 8) -- (16, 8);
    \draw (4, 0) -- (4, 8);
    \draw (0, 4) -- (8, 4);
    \draw (4, 6) -- (8, 6);
    \draw (6, 4) -- (6, 8);
    \draw (8, 12) -- (16, 12);
    \draw (12, 8) -- (12, 16);
}

\begin{document}

\begin{animateinline}[%
        controls={step},
        buttonsize=10pt
]{1}
\multiframe{4}{i=1+1}{%
    \begin{tikzpicture}
        \drawpoints
        \drawquadtree
        \fill [red] (B) circle (3pt)
        \foreach \point in {A, C, D, E, F, G, H}
            \fill [black] (\point) circle (3pt);

        \ifthenelse{\i > 1 }{%
            \begin{pgfonlayer}{background}
                \fill[orange!20] (4,6) -- (6,6) -- (6,8) -- (4,8) -- cycle;
            \end{pgfonlayer}
        }

        \ifthenelse{\i = 2}{%
            \begin{pgfonlayer}{background}
                \fill[blue!20] (6,2) -- (8,2) -- (8,4) -- (6,4) -- cycle;
                \fill[blue!20] (8,4) -- (10,4) -- (10,6) -- (8,6) -- cycle;
                \fill[blue!20] (8,6) -- (8,8) -- (6,8) -- (6,6) -- cycle;
                \fill[blue!20] (6,4) -- (4,4) -- (4,6) -- (6,6) -- cycle;
            \end{pgfonlayer}
        }

        \ifthenelse{\i = 3}{%
            \begin{pgfonlayer}{background}
                \fill[blue!20] (6,2) -- (8,2) -- (8,4) -- (6,4) -- cycle;
                \fill[blue!20] (8,4) -- (10,4) -- (10,6) -- (8,6) -- cycle;
                \fill[blue!20] (8,6) -- (8,8) -- (6,8) -- (6,6) -- cycle;
                \fill[blue!20] (6,4) -- (4,4) -- (4,6) -- (6,6) -- cycle;
            \end{pgfonlayer}
        }

    \end{tikzpicture}
}
\end{animateinline}

\end{document}
