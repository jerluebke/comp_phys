%! TEX program = lualatex
\documentclass{beamer}
\usepackage{animate}
\usepackage{graphicx}
\graphicspath{{D:/source/repos/computational_physics/project_1/python/data/}}
\usepackage{ifthen}
\usepackage{tikz}
\usetikzlibrary{backgrounds,graphs,graphdrawing,shapes.geometric}
\usegdlibrary{trees}

\newcommand{\nodelabel}[1]{%
    {\LARGE{\textcolor{blue}{$#1$}}}
}

\newcommand{\drawpoints}{%
    \draw (0, 0) rectangle (16, 16);
    \coordinate (A) at (5.5, 2.5);
    \coordinate (B) at (6.5, 5.5);
    \coordinate (C) at (5.5, 6.5);
    \coordinate (D) at (7.5, 7.5);
    \coordinate (E) at (9.5, 6.5);
    \coordinate (F) at (8.5, 8.5);
    \coordinate (G) at (12.5, 9.5);
    \coordinate (H) at (9.5, 12.5);
}

\newcommand{\drawquadtree}{%
    \draw (8, 0) -- (8, 16);
    \draw (0, 8) -- (16, 8);
    \draw (4, 0) -- (4, 8);
    \draw (0, 4) -- (8, 4);
    \draw (4, 6) -- (8, 6);
    \draw (6, 4) -- (6, 8);
    \draw (8, 12) -- (16, 12);
    \draw (12, 8) -- (12, 16);
}

\tikzstyle{rn} = [% root node
    rectangle,
    rounded corners,
    draw,
    fill=gray!30
]

\tikzstyle{leaf} = [%
    circle,
    draw,
    fill=green!30
]

\tikzstyle{branch} = [%
    circle,
    draw,
    fill=gray!30
]

\tikzstyle{null} = [%
    ellipse,
    dashed,
    draw
]

\tikzstyle{query} = [%
    circle,
    draw,
    fill=red!30
]

\tikzstyle{neighbour} = [%
    circle,
    draw,
    fill=blue!30
]

\tikzstyle{candidate} = [%
    circle,
    draw,
    dashed,
    fill=blue!20
]


\begin{document}

\begin{frame}
    \frametitle{Morton curve}
    \textit{TODO: insert image from wikipedia}
\end{frame}


\begin{frame}
    \resizebox{\textwidth}{!}{%
        \begin{tikzpicture}[scale=\textwidth/12cm]
    \def\A{\LARGE{\textcolor{blue}{$(5, 2)$}}}
    \def\B{\LARGE{\textcolor{blue}{$(6, 5)$}}}
    \def\C{\LARGE{\textcolor{blue}{$(5, 6)$}}}
    \def\D{\LARGE{\textcolor{blue}{$(7, 7)$}}}
    \def\E{\LARGE{\textcolor{blue}{$(9, 6)$}}}
    \def\F{\LARGE{\textcolor{blue}{$(8, 8)$}}}
    \def\G{\LARGE{\textcolor{blue}{$(12, 9)$}}}
    \def\H{\LARGE{\textcolor{blue}{$(9, 12)$}}}

    % \draw[step=1cm,gray,thin] (0, 0) grid (16, 16);
    \draw (0, 0) rectangle (16, 16);

    \coordinate [label=left:\A] (A) at (5.5, 2.5);
    \coordinate [label=right:\B] (B) at (6.5, 5.5);
    \coordinate [label=left:\C] (C) at (5.5, 6.5);
    \coordinate [label=below:\D] (D) at (7.5, 7.5);
    \coordinate [label=right:\E] (E) at (9.5, 6.5);
    \coordinate [label=above:\F] (F) at (8.5, 8.5);
    \coordinate [label=right:\G] (G) at (12.5, 9.5);
    \coordinate [label=above:\H] (H) at (9.5, 12.5);

    \draw (A) -- (B);
    \draw (B) -- (C);
    \draw (C) -- (D);
    \draw (D) -- (E);
    \draw (E) -- (F);
    \draw (F) -- (G);
    \draw (G) -- (H);

    \foreach \point in {A, B, C, D, E, F, G, H}
        \fill [black] (\point) circle (3pt);
\end{tikzpicture}

% vim: set ff=unix tw=79 sw=4 ts=4 et ic ai :

    }
\end{frame}


\begin{frame}
    \resizebox{\textwidth}{!}{%
        \begin{tikzpicture}
    \def\A{\LARGE{\textcolor{blue}{$A$}}}
    \def\B{\LARGE{\textcolor{blue}{$B$}}}
    \def\C{\LARGE{\textcolor{blue}{$C$}}}
    \def\D{\LARGE{\textcolor{blue}{$D$}}}
    \def\E{\LARGE{\textcolor{blue}{$E$}}}
    \def\F{\LARGE{\textcolor{blue}{$F$}}}
    \def\G{\LARGE{\textcolor{blue}{$G$}}}
    \def\H{\LARGE{\textcolor{blue}{$H$}}}

    \draw (0, 0) rectangle (16, 16);

    \coordinate [label=left:\A] (A) at (5.5, 2.5);
    \coordinate [label=right:\B] (B) at (6.5, 5.5);
    \coordinate [label=left:\C] (C) at (5.5, 6.5);
    \coordinate [label=below:\D] (D) at (7.5, 7.5);
    \coordinate [label=right:\E] (E) at (9.5, 6.5);
    \coordinate [label=above:\F] (F) at (8.5, 8.5);
    \coordinate [label=right:\G] (G) at (12.5, 9.5);
    \coordinate [label=above:\H] (H) at (9.5, 12.5);

    \draw (8, 0) -- (8, 16);
    \draw (0, 8) -- (16, 8);
    \draw (4, 0) -- (4, 8);
    \draw (0, 4) -- (8, 4);
    \draw (4, 6) -- (8, 6);
    \draw (6, 4) -- (6, 8);
    \draw (8, 12) -- (16, 12);
    \draw (12, 8) -- (12, 16);

    \foreach \point in {A, B, C, D, E, F, G, H}
        \fill [black] (\point) circle (3pt);
\end{tikzpicture}

% vim: set ff=unix tw=79 sw=4 ts=4 et ic ai :

    }
\end{frame}


\begin{frame}
    \resizebox{\textwidth}{!}{%
        \begin{tikzpicture} [tree layout]
    \graph {%
        ROOT [rn] -- {[fresh nodes]%
            / [branch] -- {%
                NULL [null], A [leaf], NULL [null], / [branch] -- {%
                    NULL [null], B [leaf], C [leaf], D [leaf]
                }
            }, E [leaf], NULL [null], / [branch] -- {%
                F [leaf], G [leaf], H [leaf], NULL [null]
            }
        }
    };
\end{tikzpicture}

    }
\end{frame}


\begin{frame}
    \resizebox{\textwidth}{!}{%
        \begin{tikzpicture} [tree layout]
    \graph {%
        ROOT [rn] -- {[fresh nodes]%
            00 [branch] -- {%
                01 [leaf], 11 [branch] -- {%
                    01 [leaf], 10 [leaf], 11 [leaf]
                }
            }, 01 [leaf], 11 [branch] -- {%
                00 [leaf], 01 [leaf], 10 [leaf]
            }
        }
    };
\end{tikzpicture}

    }
\end{frame}


\begin{frame}
\begin{animateinline}[%
    controls={step},
    buttonsize=10pt
]{1}
    \multiframe{4}{i=1+1}{%
        \resizebox{\textwidth}{!}{%
            \begin{tikzpicture}
    \drawpoints

    \ifthenelse{\i > 0}{%
        \draw (8, 0) -- (8, 16);
        \draw (0, 8) -- (16, 8);
    }{}

    \ifthenelse{\i = 1}{%
        \coordinate [label=45:\nodelabel{00}] (00) at (0, 0);
        \coordinate [label=45:\nodelabel{01}] (01) at (8, 0);
        \coordinate [label=315:\nodelabel{10}] (10) at (0, 16);
        \coordinate [label=315:\nodelabel{11}] (11) at (8, 16);
    }{}

    \ifthenelse{\i > 1}{%
        \draw (4, 0) -- (4, 8);
        \draw (0, 4) -- (8, 4);
        \draw (8, 12) -- (16, 12);
        \draw (12, 8) -- (12, 16);
    }{}

    \ifthenelse{\i = 2}{%
        \begin{pgfonlayer}{background}
            \fill[orange!20] (0,0) -- (8,0) -- (8,8) -- (0,8) -- cycle;
            \fill[orange!20] (8,8) -- (16,8) -- (16,16) -- (8,16) -- cycle;
        \end{pgfonlayer}
        \coordinate [label=45:\nodelabel{0000}] (0000) at (0, 0);
        \coordinate [label=45:\nodelabel{0001}] (0001) at (4, 0);
        \coordinate [label=45:\nodelabel{0010}] (0010) at (0, 4);
        \coordinate [label=45:\nodelabel{0011}] (0011) at (4, 4);
        \coordinate [label=315:\nodelabel{1100}] (1100) at (8, 12);
        \coordinate [label=315:\nodelabel{1101}] (1101) at (12, 12);
        \coordinate [label=315:\nodelabel{1110}] (1110) at (8, 16);
        \coordinate [label=315:\nodelabel{1111}] (1111) at (12, 16);
    }{}

    \ifthenelse{\i > 2}{%
        \draw (4, 6) -- (8, 6);
        \draw (6, 4) -- (6, 8);
    }{}

    \ifthenelse{\i = 3}{%
        \begin{pgfonlayer}{background}
            \fill[orange!20] (4,4) -- (8,4) -- (8,8) -- (4,8) -- cycle;
        \end{pgfonlayer}
        \coordinate [label=315:\nodelabel{001100}] (001100) at (4, 4);
        \coordinate [label=315:\nodelabel{001101}] (001101) at (6, 4);
        \coordinate [label=45:\nodelabel{001110}] (001110) at (4, 8);
        \coordinate [label=45:\nodelabel{001111}] (001111) at (6, 8);
    }{}

    \foreach \point in {A, B, C, D, E, F, G, H}
        \fill [black] (\point) circle (3pt);

    \ifthenelse{\i = 4}{%
        \begin{pgfonlayer}{background}
            \fill[orange!20] (4,6) -- (6,6) -- (6,8) -- (4,8) -- cycle;
        \end{pgfonlayer}

        \draw (A) -- (B);
        \draw (B) -- (C);
        \draw (C) -- (D);
        \draw (D) -- (E);
        \draw (E) -- (F);
        \draw (F) -- (G);
        \draw (G) -- (H);

        \coordinate [label=left:\nodelabel{(5,6)}] (mortonC) at (5.5, 6.5);
        \coordinate [label=45:\nodelabel{00111001}] (pointC) at (4, 8);
        \fill [red] (C) circle (3pt);
    }{}

\end{tikzpicture}

        }
    }
\end{animateinline}
\end{frame}


\begin{frame}
    \frametitle{space seperation animation}
    \begin{center}
        \animategraphics[%
            draft,
            % every=3,
            type=png,
            width=.99 \textwidth,
            controls={play,step,stop},
            buttonsize=10pt
        ]{3}{plots/out-}{000}{064}
    \end{center}
\end{frame}


\begin{frame}
    \frametitle{quadtree construction}
    \begin{center}
        \animategraphics[%
            draft,
            % every=3,
            type=png,
            width=\textwidth,
            controls={play,step,stop},
            buttonsize=10pt
        ]{3}{graphs/extent_out-}{001}{063}
    \end{center}
\end{frame}


\begin{frame}
\begin{animateinline}[%
    controls={step},
    buttonsize=10pt
]{1}
    \multiframe{4}{i=1+1}{%
        \resizebox{\textwidth}{!}{%
            \documentclass{standalone}
\usepackage{animate}
\usepackage{graphicx}
\usepackage{ifthen}
\usepackage{tikz}
\usetikzlibrary{backgrounds}

\newcommand{\nodelabel}[1]{%
    {\LARGE{\textcolor{blue}{$#1$}}}
}

\newcommand{\drawpoints}{%
    \draw (0, 0) rectangle (16, 16);
    \coordinate (A) at (5.5, 2.5);
    \coordinate (B) at (6.5, 5.5);
    \coordinate (C) at (5.5, 6.5);
    \coordinate (D) at (7.5, 7.5);
    \coordinate (E) at (9.5, 6.5);
    \coordinate (F) at (8.5, 8.5);
    \coordinate (G) at (12.5, 9.5);
    \coordinate (H) at (9.5, 12.5);
}

\newcommand{\drawquadtree}{%
    \draw (8, 0) -- (8, 16);
    \draw (0, 8) -- (16, 8);
    \draw (4, 0) -- (4, 8);
    \draw (0, 4) -- (8, 4);
    \draw (4, 6) -- (8, 6);
    \draw (6, 4) -- (6, 8);
    \draw (8, 12) -- (16, 12);
    \draw (12, 8) -- (12, 16);
}

\begin{document}

\begin{animateinline}[%
        controls={step},
        buttonsize=10pt
]{1}
\multiframe{4}{i=1+1}{%
    \begin{tikzpicture}
        \drawpoints
        \drawquadtree
        \fill [red] (B) circle (3pt);
        \foreach \point in {A, C, D, E, F, G, H}
            \fill [black] (\point) circle (3pt);

        \ifthenelse{4 > \i > 1 }{%
            \begin{pgfonlayer}{background}
                \fill[orange!20] (6,4) -- (6,6) -- (8,6) -- (8,4) -- cycle;
            \end{pgfonlayer}
        }

        \ifthenelse{\i = 2}{%
            \begin{pgfonlayer}{background}
                \fill[blue!20] (6,2) -- (8,2) -- (8,4) -- (6,4) -- cycle;
                \fill[blue!20] (8,4) -- (10,4) -- (10,6) -- (8,6) -- cycle;
                \fill[blue!20] (8,6) -- (8,8) -- (6,8) -- (6,6) -- cycle;
                \fill[blue!20] (6,4) -- (4,4) -- (4,6) -- (6,6) -- cycle;
            \end{pgfonlayer}
        }

        \ifthenelse{\i = 3}{%
            \begin{pgfonlayer}{background}
                \fill[blue!20] (8,6) -- (8,8) -- (6,8) -- (6,6) -- cycle;
                \fill[blue!20] (4,0) -- (8,0) -- (8,4) -- (4,4) -- cycle;
                \fill[blue!20] (8,0) -- (16,0) -- (16,8) -- (8,8) -- cycle;
            \end{pgfonlayer}
        }

        \ifthenelse{\i = 4}{%
            \coordinate [label=315:\nodelabel{(6,5)}] (pointB) at (6.5, 5.5);
            \coordinate [label=right:\nodelabel{(5,2)}] (pointA) at (5.5, 2.5);
            \coordinate [label=225:\nodelabel{(7,7)}] (pointD) at (7.5, 7.5);
            \coordinate [label=right:\nodelabel{(9,6)}] (pointE) at (9.5, 6.5);
            \fill[blue] (A) circle (3pt);
            \fill[blue] (D) circle (3pt);
            \fill[blue] (E) circle (3pt);
        }

    \end{tikzpicture}
}
\end{animateinline}

\end{document}

        }
    }
\end{animateinline}
\end{frame}


\begin{frame}
\begin{animateinline}[%
    controls={step},
    buttonsize=10pt
]{1}
    \multiframe{4}{i=1+1}{%
        \resizebox{\textwidth}{!}{%
            \documentclass{standalone}
\usepackage{animate}
\usepackage{graphicx}
\usepackage{ifthen}
\usepackage{tikz}
\usetikzlibrary{backgrounds}

\newcommand{\nodelabel}[1]{%
    {\LARGE{\textcolor{blue}{$#1$}}}
}

\newcommand{\drawpoints}{%
    \draw (0, 0) rectangle (16, 16);
    \coordinate (A) at (5.5, 2.5);
    \coordinate (B) at (6.5, 5.5);
    \coordinate (C) at (5.5, 6.5);
    \coordinate (D) at (7.5, 7.5);
    \coordinate (E) at (9.5, 6.5);
    \coordinate (F) at (8.5, 8.5);
    \coordinate (G) at (12.5, 9.5);
    \coordinate (H) at (9.5, 12.5);
}

\newcommand{\drawquadtree}{%
    \draw (8, 0) -- (8, 16);
    \draw (0, 8) -- (16, 8);
    \draw (4, 0) -- (4, 8);
    \draw (0, 4) -- (8, 4);
    \draw (4, 6) -- (8, 6);
    \draw (6, 4) -- (6, 8);
    \draw (8, 12) -- (16, 12);
    \draw (12, 8) -- (12, 16);
}

\begin{document}

\begin{animateinline}[%
        controls={step},
        buttonsize=10pt
]{1}
\multiframe{4}{i=1+1}{%
    \begin{tikzpicture}
        \drawpoints
        \drawquadtree
        \fill [red] (E) circle (3pt);
        \foreach \point in {A, B, C, D, F, G, H}
            \fill [black] (\point) circle (3pt);

        \ifthenelse{4 > \i > 1 }{%
            \begin{pgfonlayer}{background}
                \fill[orange!20] (8,0) -- (16,0) -- (16,8) -- (8,8) -- cycle;
            \end{pgfonlayer}
        }

        \ifthenelse{\i = 2}{%
            \begin{pgfonlayer}{background}
                \fill[blue!20] (0,0) -- (8,0) -- (8,8) -- (0,8) -- cycle;
                \fill[blue!20] (8,8) -- (16,8) -- (16,16) -- (8,16) -- cycle;
            \end{pgfonlayer}
        }

        \ifthenelse{\i = 3}{%
            \begin{pgfonlayer}{background}
                \fill[blue!20] (4,0) -- (8,0) -- (8,4) -- (4,4) -- cycle;
                \fill[blue!20] (6,4) -- (8,4) -- (8,8) -- (6,8) -- cycle;
                \fill[blue!20] (8,8) -- (16,8) -- (16,12) -- (8,12) -- cycle;
            \end{pgfonlayer}
        }

        \ifthenelse{\i = 4}{%
            \coordinate [label=right:\nodelabel{(9,6)}] (pointE) at (9.5, 6.5);
            \coordinate [label=right:\nodelabel{(5,2)}] (pointA) at (5.5, 2.5);
            \coordinate [label=315:\nodelabel{(6,5)}] (pointB) at (6.5, 5.5);
            \coordinate [label=225:\nodelabel{(7,7)}] (pointD) at (7.5, 7.5);
            \coordinate [label=45:\nodelabel{(8,8)}] (pointF) at (8.5, 8.5);
            \coordinate [label=right:\nodelabel{(12,9)}] (pointG) at (12.5, 9.5);
            \fill[blue] (A) circle (3pt);
            \fill[blue] (B) circle (3pt);
            \fill[blue] (D) circle (3pt);
            \fill[blue] (F) circle (3pt);
            \fill[blue] (G) circle (3pt);
        }

    \end{tikzpicture}
}
\end{animateinline}

\end{document}

        }
    }
\end{animateinline}
\end{frame}


\begin{frame}
\begin{tikzpicture} [tree layout]
\graph {%
    ROOT [rn] -- {[fresh nodes]
        / [branch] -- {A [branch] -- {/ [candidate]}, / [branch] -- {%
                / [candidate], B [query], C [branch], D [candidate]
            }
        },
        E [branch] -- {/ [branch] -- {/ [candidate]}}, / [branch] -- {%
            F [branch], G [branch], H [branch]
        }
    }
};
\end{tikzpicture}
\end{frame}


\begin{frame}
\begin{tikzpicture} [tree layout]
\graph {%
    ROOT [rn] -- {[fresh nodes]
        / [branch] -- {A [neighbour]}, / [branch] -- {%
            B [query], C [branch], D [neighbour]
        },
        E [neighbour], / [branch] -- {%
            F [branch], G [branch], H [branch]
        }
    }
};
\end{tikzpicture}
\end{frame}


\begin{frame}
\begin{tikzpicture} [tree layout]
\graph {%
    ROOT [rn] -- {[fresh nodes]
        / [candidate] -- {A [branch], / [branch] -- {%
                B [branch], C [branch], D [branch]
            }
        },
        E [query], / [candidate] -- {%
            F [branch], G [branch], H [branch]
        }
    }
};
\end{tikzpicture}
\end{frame}


\begin{frame}
\begin{tikzpicture} [tree layout]
\graph {%
    ROOT [rn] -- {[fresh nodes]
        / [branch] -- {A [neighbour]}, / [branch] -- {%
            B [neighbour], C [branch], D [neighbour]
        },
        E [query], / [branch] -- {%
            F [neighbour], G [neighbour], H [branch]
        }
    }
};
\end{tikzpicture}
\end{frame}

\end{document}

% vim: set ff=unix tw=79 sw=4 ts=4 et ic ai :
