%! TEX program = lualatex
\documentclass[mathserif]{beamer}
\usepackage{amsmath}
\usepackage{animate}
\usepackage{graphicx}
    \graphicspath{{D:/source/repos/computational_physics/project_1/latex/img/}}
\usepackage{ifthen}
\usepackage[outputdir=latex_build]{minted}
    \usemintedstyle{friendly}
\usepackage{tikz}
\usetikzlibrary{backgrounds,graphs,graphdrawing,shapes.geometric}
\usegdlibrary{trees}

\newcommand{\nodelabel}[1]{%
    {\LARGE{\textcolor{blue}{$#1$}}}
}

\newcommand{\drawpoints}{%
    \draw (0, 0) rectangle (16, 16);
    \coordinate (A) at (5.5, 2.5);
    \coordinate (B) at (6.5, 5.5);
    \coordinate (C) at (5.5, 6.5);
    \coordinate (D) at (7.5, 7.5);
    \coordinate (E) at (9.5, 6.5);
    \coordinate (F) at (8.5, 8.5);
    \coordinate (G) at (12.5, 9.5);
    \coordinate (H) at (9.5, 12.5);
}

\newcommand{\drawquadtree}{%
    \draw (8, 0) -- (8, 16);
    \draw (0, 8) -- (16, 8);
    \draw (4, 0) -- (4, 8);
    \draw (0, 4) -- (8, 4);
    \draw (4, 6) -- (8, 6);
    \draw (6, 4) -- (6, 8);
    \draw (8, 12) -- (16, 12);
    \draw (12, 8) -- (12, 16);
}

\tikzstyle{rn} = [% root node
    rectangle,
    rounded corners,
    draw,
    fill=gray!30
]

\tikzstyle{leaf} = [%
    circle,
    draw,
    fill=green!30
]

\tikzstyle{branch} = [%
    circle,
    draw,
    % fill=gray!30
]

\tikzstyle{null} = [%
    ellipse,
    dashed,
    draw
]

\tikzstyle{query} = [%
    circle,
    draw,
    fill=red!30
]

\tikzstyle{neighbour} = [%
    circle,
    draw,
    fill=blue!30
]

\tikzstyle{candidate} = [%
    circle,
    draw,
    dashed,
    fill=blue!20
]


\setcounter{section}{-1}
\renewcommand{\theequation}{\arabic{section}.\arabic{equation}}
\newcommand{\bb}{0\textrm{b}}
\newcommand{\xx}{0\textrm{x}}
\newcommand{\AND}{\textrm{\textbf{ \& }}}
\newcommand{\OR}{\textrm{\textbf{\textbar }}}
\newcommand{\XOR}{\textrm{\textbf{ \^{} }}}


\title{Nearest Neighbour Search using Quadtrees}
\author{Jeremiah Lübke}
\date{\today}
\subject{Data Structures and Algorithms}


\begin{document}

\frame{\titlepage}

\begin{frame}
    \frametitle{Table of Contents}
    \tableofcontents
\end{frame}

\begin{frame}
    \section{Motivation}
    \begin{columns}[T]
    \column{.5\textwidth}
    \subsection*{Applications}
    \emph{TODO...}
    \column{.5\textwidth}
    \subsection*{Algorithmic aspects}
    \emph{TODO...}
    \end{columns}
\end{frame}


\input{slides/morton.tex}

\section{Quadtrees}
\subsection{Introduction}
\begin{frame}
    \frametitle{Quadtrees}
    Divide some given space according to its \emph{particle density}
    \begin{figure}
        \centering
        \resizebox{\textwidth}{!}{%
            \section{Quadtrees}
\subsection{Introduction}
\begin{frame}
    \frametitle{Quadtrees}
    Divide some given space according to its \emph{particle density}
    \begin{figure}
        \centering
        \resizebox{\textwidth}{!}{%
            \section{Quadtrees}
\subsection{Introduction}
\begin{frame}
    \frametitle{Quadtrees}
    Divide some given space according to its \emph{particle density}
    \begin{figure}
        \centering
        \resizebox{\textwidth}{!}{%
            \input{tikz/quadtree.tex}
        }
    \end{figure}
\end{frame}

\begin{frame}
    \begin{figure}
        \centering
        \resizebox{.8\textwidth}{!}{%
            \input{tikz/spatial_part.tex}
        }
        \caption{spatial partition}
    \end{figure}
\end{frame}

\subsection{Quadtree \& Morton Curve}
\begin{frame}
    \frametitle{Quadtree \& Morton Curve}
    \textbf{Properties:}
    \begin{itemize}
        \item The four children of a given a node are numbered consecutively
            $0,\dots,3$ \\
        \item When stepping through the levels of the tree towards a given
            point, the numbers of the nodes add up to the points morton key
            (possibly truncated) \\
        \item Points with a common parent node are neighbouring on the morton
            curve
    \end{itemize}
\end{frame}

\begin{frame}
    \begin{figure}
        \centering
        \caption{key = 0b001110}
        \resizebox{.6\textwidth}{!}{%
            \input{tikz/numbered_tree.tex}
        }
    \end{figure}

    \noindent\rule{\textwidth}{1pt}

    \textbf{Notes:}
    \begin{itemize}
        \item max depth of tree $=\frac{\textrm{key length}}{\textrm{dim}}$ \\
        \item Resolution of domain partitioned by the tree $=2^{\textrm{tree
            depth}}-1$
    \end{itemize}
\end{frame}

\begin{frame}
    \begin{figure}
        \centering
        \begin{animateinline}[controls={step,play,stop},buttonsize=10pt]{.5}
            \multiframe{4}{i=1+1}{%
                \resizebox{.7\textwidth}{!}{%
                    \input{tikz/morton_anim.tex}
                }
            }
        \end{animateinline}
        \caption{Connection between keys of node and point}
    \end{figure}
\end{frame}

\subsection{Example}
\begin{frame}
    \textbf{Space Seperation Animation}
    \begin{center}
        \animategraphics[%
            draft,
            % every=3,
            type=png,
            width=.99 \textwidth,
            controls={play,step,stop},
            buttonsize=10pt
        ]{3}{plots/out-}{000}{064}
    \end{center}
\end{frame}


\begin{frame}
    \textbf{Quadtree Construction}
    \begin{center}
        \animategraphics[%
            draft,
            % every=3,
            type=png,
            width=\textwidth,
            controls={play,step,stop},
            buttonsize=10pt
        ]{3}{graphs/extent_out-}{001}{063}
    \end{center}
\end{frame}

        }
    \end{figure}
\end{frame}

\begin{frame}
    \begin{figure}
        \centering
        \resizebox{.8\textwidth}{!}{%
            \input{tikz/spatial_part.tex}
        }
        \caption{spatial partition}
    \end{figure}
\end{frame}

\subsection{Quadtree \& Morton Curve}
\begin{frame}
    \frametitle{Quadtree \& Morton Curve}
    \textbf{Properties:}
    \begin{itemize}
        \item The four children of a given a node are numbered consecutively
            $0,\dots,3$ \\
        \item When stepping through the levels of the tree towards a given
            point, the numbers of the nodes add up to the points morton key
            (possibly truncated) \\
        \item Points with a common parent node are neighbouring on the morton
            curve
    \end{itemize}
\end{frame}

\begin{frame}
    \begin{figure}
        \centering
        \caption{key = 0b001110}
        \resizebox{.6\textwidth}{!}{%
            \input{tikz/numbered_tree.tex}
        }
    \end{figure}

    \noindent\rule{\textwidth}{1pt}

    \textbf{Notes:}
    \begin{itemize}
        \item max depth of tree $=\frac{\textrm{key length}}{\textrm{dim}}$ \\
        \item Resolution of domain partitioned by the tree $=2^{\textrm{tree
            depth}}-1$
    \end{itemize}
\end{frame}

\begin{frame}
    \begin{figure}
        \centering
        \begin{animateinline}[controls={step,play,stop},buttonsize=10pt]{.5}
            \multiframe{4}{i=1+1}{%
                \resizebox{.7\textwidth}{!}{%
                    \input{tikz/morton_anim.tex}
                }
            }
        \end{animateinline}
        \caption{Connection between keys of node and point}
    \end{figure}
\end{frame}

\subsection{Example}
\begin{frame}
    \textbf{Space Seperation Animation}
    \begin{center}
        \animategraphics[%
            draft,
            % every=3,
            type=png,
            width=.99 \textwidth,
            controls={play,step,stop},
            buttonsize=10pt
        ]{3}{plots/out-}{000}{064}
    \end{center}
\end{frame}


\begin{frame}
    \textbf{Quadtree Construction}
    \begin{center}
        \animategraphics[%
            draft,
            % every=3,
            type=png,
            width=\textwidth,
            controls={play,step,stop},
            buttonsize=10pt
        ]{3}{graphs/extent_out-}{001}{063}
    \end{center}
\end{frame}

        }
    \end{figure}
\end{frame}

\begin{frame}
    \begin{figure}
        \centering
        \resizebox{.8\textwidth}{!}{%
            \input{tikz/spatial_part.tex}
        }
        \caption{spatial partition}
    \end{figure}
\end{frame}

\subsection{Quadtree \& Morton Curve}
\begin{frame}
    \frametitle{Quadtree \& Morton Curve}
    \textbf{Properties:}
    \begin{itemize}
        \item The four children of a given a node are numbered consecutively
            $0,\dots,3$ \\
        \item When stepping through the levels of the tree towards a given
            point, the numbers of the nodes add up to the points morton key
            (possibly truncated) \\
        \item Points with a common parent node are neighbouring on the morton
            curve
    \end{itemize}
\end{frame}

\begin{frame}
    \begin{figure}
        \centering
        \caption{key = 0b001110}
        \resizebox{.6\textwidth}{!}{%
            \input{tikz/numbered_tree.tex}
        }
    \end{figure}

    \noindent\rule{\textwidth}{1pt}

    \textbf{Notes:}
    \begin{itemize}
        \item max depth of tree $=\frac{\textrm{key length}}{\textrm{dim}}$ \\
        \item Resolution of domain partitioned by the tree $=2^{\textrm{tree
            depth}}-1$
    \end{itemize}
\end{frame}

\begin{frame}
    \begin{figure}
        \centering
        \begin{animateinline}[controls={step,play,stop},buttonsize=10pt]{.5}
            \multiframe{4}{i=1+1}{%
                \resizebox{.7\textwidth}{!}{%
                    \input{tikz/morton_anim.tex}
                }
            }
        \end{animateinline}
        \caption{Connection between keys of node and point}
    \end{figure}
\end{frame}

\subsection{Example}
\begin{frame}
    \textbf{Space Seperation Animation}
    \begin{center}
        \animategraphics[%
            draft,
            % every=3,
            type=png,
            width=.99 \textwidth,
            controls={play,step,stop},
            buttonsize=10pt
        ]{3}{plots/out-}{000}{064}
    \end{center}
\end{frame}


\begin{frame}
    \textbf{Quadtree Construction}
    \begin{center}
        \animategraphics[%
            draft,
            % every=3,
            type=png,
            width=\textwidth,
            controls={play,step,stop},
            buttonsize=10pt
        ]{3}{graphs/extent_out-}{001}{063}
    \end{center}
\end{frame}


\setcounter{equation}{0}
\section{Implementation}
\subsection{Structures}
\begin{frame}[fragile]
    \frametitle{Implementation}
    \textbf{Structures}
    \begin{minted}[linenos,tabsize=4]{c}
struct node_t {
    uint16_t key;
    item_t *item;
    uint8_t level;
    node_t *children[];
};

struct item_t {
    uint16_t key;
    /* content of item...
     * e.g. coordinates, mass, size, etc. */
    bool is_last;
};
    \end{minted}
\end{frame}

\begin{frame}
    \textbf{Notes:}
    \begin{itemize}
        \item struct -- summary of arbitrary datatypes to a new type \\
        \item uint16\_t -- type unsigned int, 16 bit wide \\
        \item *something -- adresse of something (pointer) \\
        \item *something[] -- array of pointers
        \item accessesing members of a struct: struct.member,
            struct\_ptr-->member
    \end{itemize}
\end{frame}


\subsection{Inserting}
\begin{frame}
    \textbf{Inserting new values - the idea}
    \begin{itemize}
        \item Items with common parent node are neighbours on the morton curve
            $\implies$ calculate parent node of current and next item \\
        \item If that parent node does not exist, build all nodes - starting from
            the last existing one - including the parent node plus one child
            for the current item
    \end{itemize}
    \textbf{Advantage:} Fast building of the tree without reinserting items
\end{frame}

\begin{frame}
    \textbf{How does one calculate the parent node?} \\
    When comparing two keys, we are interested in the position of the first
    different bit. This gives the level at which the two items do not share
    the same branch anymore.

    \noindent\rule{\textwidth}{1pt}

    \textbf{Example:} item[0].coords $= (5, 6)$, item[1].coords $= (6, 5)$ \\
    \begin{align}
        \begin{split}
            w = \textrm{item[0].key} \XOR \textrm{item[1].key} =& \bb00111001 \\
            \XOR& \bb00110101 \\
            =& \bb00001100 = 12
        \end{split}
    \end{align}
    \begin{equation}
        \textrm{level of common parent} =
        \textrm{maxlevel}-\frac{\lfloor \log_2\left(w\right)
        \rfloor}{\textrm{dim}} = 2
    \end{equation}
    {\small\textit{(here: maxlevel=4, dim=2)}}
\end{frame}

\begin{frame}[fragile]
    \textbf{Inserting new values - the code}
    \begin{minted}[linenos,breaklines,tabsize=4]{c}
uint8_t insert( node_t *head, item_t items[] )
{
    uint8_t new_levels;
    uint16_t significant_bit, level_of_common_parent;
    significant_bit = bits_at_position( items[0].key, head->level+1 );

    if ( head->children[significant_bit] != NULL ) {
        return insert( head->children[significant_bit], items );
    }
    \end{minted}
\end{frame}

\begin{frame}[fragile]
    \textit{cont\dots}
    \begin{minted}[linenos,firstnumber=10,breaklines,tabsize=4]{c}
    else {
        if ( items[0].is_last )
            level_of_common_parent = 0;
        else
            level_of_common_parent = \
                maxlevel - log( items[0].key ^ items[1].key ) / 2;

        new_levels = level_of_common_parent - head->level;
        if ( new_levels <= 0 )
            new_levels = 1;
        head->children[significant_bit] = \
            build_branch( head->level+1, new_levels, &items[0] );

        return new_levels;
    }
}
    \end{minted}
\end{frame}


\subsection{Searching}
\begin{frame}[fragile]
    \textbf{Searching for an item}
    \begin{minted}[linenos,tabsize=4]{c}
node_t *search( uint16_t key, node_t *head )
{
    uint16_t significant_bit;
    significant_bit = bits_at_position( key, head->level+1 );
    while ( head->children != NULL
            && head->children[significant_bit] != NULL
            && key != head->key ) {
        head = head->children[significant_bit];
        significant_bit = bits_at_position( key, head->level+1 );
    }
    return head;
}
    \end{minted}
\end{frame}


\section{Nearest Neighbours}
\begin{frame}
    \frametitle{Nearest Neighbours}
    \textbf{putting it all together\dots}
    \begin{itemize}
        \item for each item: calculate morton key \\
        \item for a given query point: calculate neighbour candidates on morton
            curve \\
        \item for each candidate: \\
        \begin{itemize}
            \item is the candidate on a boundary? If yes, continue with next \\
            \item search the candidate's node in the tree $\implies$ 3 cases \\
            \begin{enumerate}
                \item candidate exists and is end node \textrightarrow take as
                    neighbour and continue \\
                \item candidate doesn't exist \textrightarrow take last
                    existing parent node as neighbour and continue \\
                \item candidate exists and has further children \textrightarrow
                    search its children recursively, facing in the direction of
                    the query point
            \end{enumerate}
        \end{itemize}
    \end{itemize}
\end{frame}

\begin{frame}
    
\end{frame}




% \begin{frame}
%     \frametitle{space seperation animation}
%     \begin{center}
%         \animategraphics[%
%             draft,
%             % every=3,
%             type=png,
%             width=.99 \textwidth,
%             controls={play,step,stop},
%             buttonsize=10pt
%         ]{3}{plots/out-}{000}{064}
%     \end{center}
% \end{frame}

%
% \begin{frame}
%     \frametitle{quadtree construction}
%     \begin{center}
%         \animategraphics[%
%             draft,
%             % every=3,
%             type=png,
%             width=\textwidth,
%             controls={play,step,stop},
%             buttonsize=10pt
%         ]{3}{graphs/extent_out-}{001}{063}
%     \end{center}
% \end{frame}


\begin{frame}
\begin{animateinline}[%
    controls={step},
    buttonsize=10pt
]{1}
    \multiframe{4}{i=1+1}{%
        \resizebox{\textwidth}{!}{%
            \input{tikz/abb8i.tex}
        }
    }
\end{animateinline}
\end{frame}


\begin{frame}
\begin{animateinline}[%
    controls={step},
    buttonsize=10pt
]{1}
    \multiframe{4}{i=1+1}{%
        \resizebox{\textwidth}{!}{%
            \input{tikz/abb8ii.tex}
        }
    }
\end{animateinline}
\end{frame}


\begin{frame}
\begin{tikzpicture} [tree layout]
\graph {%
    ROOT [rn] -- {[fresh nodes]
        / [branch] -- {A [branch] -- {/ [candidate]}, / [branch] -- {%
                / [candidate], B [query], C [branch], D [candidate]
            }
        },
        E [branch] -- {/ [branch] -- {/ [candidate]}}, / [branch] -- {%
            F [branch], G [branch], H [branch]
        }
    }
};
\end{tikzpicture}
\end{frame}


% \begin{frame}
% \begin{tikzpicture} [tree layout]
% \graph {%
%     ROOT [rn] -- {[fresh nodes]
%         / [branch] -- {A [neighbour]}, / [branch] -- {%
%             B [query], C [branch], D [neighbour]
%         },
%         E [neighbour], / [branch] -- {%
%             F [branch], G [branch], H [branch]
%         }
%     }
% };
% \end{tikzpicture}
% \end{frame}
%
%
% \begin{frame}
% \begin{tikzpicture} [tree layout]
% \graph {%
%     ROOT [rn] -- {[fresh nodes]
%         / [candidate] -- {A [branch], / [branch] -- {%
%                 B [branch], C [branch], D [branch]
%             }
%         },
%         E [query], / [candidate] -- {%
%             F [branch], G [branch], H [branch]
%         }
%     }
% };
% \end{tikzpicture}
% \end{frame}
%
%
% \begin{frame}
% \begin{tikzpicture} [tree layout]
% \graph {%
%     ROOT [rn] -- {[fresh nodes]
%         / [branch] -- {A [neighbour]}, / [branch] -- {%
%             B [neighbour], C [branch], D [neighbour]
%         },
%         E [query], / [branch] -- {%
%             F [neighbour], G [neighbour], H [branch]
%         }
%     }
% };
% \end{tikzpicture}
% \end{frame}
%
\end{document}

% vim: set ff=unix tw=79 sw=4 ts=4 et ic ai :
