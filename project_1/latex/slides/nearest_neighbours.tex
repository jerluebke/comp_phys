\section{Nearest Neighbours}
\begin{frame}
    \frametitle{Nearest Neighbours}
    \textbf{putting it all together\dots}
    \begin{itemize}
        \item for each item: calculate morton key \\
        \item for a given query point: calculate neighbour candidates on morton
            curve \\
        \item for each candidate: \\
        \begin{itemize}
            \item is the candidate on a boundary? If yes, continue with next \\
            \item search the candidate's node in the tree $\implies$ 3 cases \\
            \begin{enumerate}
                \item candidate exists and is end node \textrightarrow take as
                    neighbour and continue \\
                \item candidate doesn't exist \textrightarrow take last
                    existing parent node as neighbour and continue \\
                \item candidate exists and has further children \textrightarrow
                    search its children recursively, facing in the direction of
                    the query point
            \end{enumerate}
        \end{itemize}
    \end{itemize}
\end{frame}

\begin{frame}
    
\end{frame}
